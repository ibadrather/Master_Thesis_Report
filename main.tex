

\documentclass[
thesis  % please do not delete or change this line!
]{csthes}

%\documentclass[12pt]{report}

% The following packages are loaded automatically. Do not load them
% manually as you might risk an option clash:
%   -babel (heading translation and hyphenation)
%   -natbib (citing, see www.ctan.org/tex-archive/macros/latex/contrib/natbib/natbib.pdf)
%   -fancyhdr (headers and footers)
%   -fontenc (output encoding)

% Set the encoding your tex-source is written in.
% Use utf8 if possible. Other options that might 
% work for you are:
%   latin1 (Unix/Linux or Windows)
%   ansinew (Windows)
%   applemac (Mac)
\usepackage[utf8]{inputenc}

%for tables
\usepackage[table,xcdraw]{xcolor}
\usepackage{makecell}
\usepackage{array}
\newcolumntype{L}{>{\centering\arraybackslash}m{3cm}}
\renewcommand\theadfont{\bfseries}


%for graphics
\usepackage{graphicx}
\usepackage{subcaption}

% for pseudo algorithms
\usepackage{algpseudocode}
\usepackage{algorithm2e}

% For the online version enable hyperlinks
\usepackage{hyperref}

\usepackage{amsmath}

\usepackage{multirow}


\setlength{\tabcolsep}{6pt}
\renewcommand{\arraystretch}{1.5}

\usepackage{rotating}

\usepackage{tabularx}
\usepackage{booktabs}

\usepackage{float}
%%%%%%%%%%%%%%%%%%%%%%%%%%%%%%%%%%%%%%%%%%%%%
%  Set information about your thesis here  %
%%%%%%%%%%%%%%%%%%%%%%%%%%%%%%%%%%%%%%%%%%%%%
\title{Digital Video Stabilization \\ of an Oscillating Camera \\
using an \\ Inertial Measurement Unit}
\subtitle{}% optional
\setthesistype{Master}
\author{Ibad Rather \matr{1532894}}

\degree{MS Mechatronics}
\supervisor{\\Company: Dr. Steffen Urban \\
            University: Prof. Dr. Ing. Bhaskar Choubey}% most likely

% Start of content
\begin{document}
% First set the title page
\maketitle
% Give a short summary of your thesis
\begin{abstract}
Video stabilization is ubiquitous in today's smartphones and action cameras. However, instead of classical “post-processing” image stabilization where image features are extracted and tracked over time [1] or modern deep learning based methods using dense optical flow [2], they need to perform the motion estimation on constrained hardware systems and with very high  frame-rates $(\geq 120 fps).$ 

Hence the motion estimation, that is necessary to correct vibrations, shaking or rolling shutter distortions is performed using Inertial Measurement Units (IMUs). Typically, the motion estimates are reduced to rotational motions [3] (using gyroscopes) as the distance between frames is small and the field-of-view of the lenses large $(\geq 50^\circ)$. In addition, estimating translational motion is a difficult challenge using low-cost IMUs as the double integration of accelerometer readings is prone to strong drift and leads to large errors even over short time periods $(\leq 1s)$. 

This thesis should investigate and develop an image stabilization method for an oscillating camera. The camera has a small field-of-view and hence small translational movements along the viewing direction will be visible and need to be stabilized. The camera is coupled with an IMU and attached to a linear stage, that will mimic the oscillating movements. The camera records a monitor or a static target. An additional goal is the implementation, optimization and testing of the proposed algorithm on a modern embedded chip.

\end{abstract}
% Set the table of contents
\tableofcontents%

% And other tables/lists optionally
\listoftables%
\listoffigures%

% Start the main work (handles page numbering etc.)
\mainmatter

% While writing your thesis it is advisable to only load the chapter you are currently working on. It decreases the compile time. 

% 1. Chapter: ''Introduction''
\chapter{Introduction} \label{chapter_one}

Cameras have become an integral part of our lives in the current world  \citet{ibad2022}. People around us use cameras regularly, be it their smartphone camera, a hobbyist camera setup or professional cameras. Also everyone want a cameras that produce the good picture quality for as cheap as possible. Researchers are constantly working towards making this technology better while also trying to make it cheaper.

There are many challenges involved in making the picture quality better. For example you may have noticed if you are taking a photograph and the camera moves while taking the photograph or a video, the photo does not look good. This problem is solvable if you have a tripod or expensive motorised stabilisers like Gimbals. But as you may have understood that is not always feasible, economically and logistically.

These problems are aggravated on microscopic levels or if you are using a zoom lens, in that case even very small movements can cause the picture quality to degrade. That is exactly the problem I am facing for this research. I am working with a GoPro Hero 10 but with a different lens than fitted from the factory one. I switched the lens to a Macro or Zoom Lens with less field of view (FOV) and different focal length. Switching the lens resulted in the default stabilising provided by the GoPro not working.

As a part of this research work, I will explore different ways in which I can stabilise the video feed from the camera. Before going into depth, I need to first talk about a lot of important things. First of all let us talk about why am I even doing this? Is this even a real problem that needs to  be solved?

% Motivation
\section{Motivation}
\begin{itemize}
\item Why are we doing this?
\item Why is this relevant?
\item Why is this exciting?
\item What concrete problem do we want to solve?
\item Is this problem big enough?
\item What makes this problem challenging?
\item Why an IMU Sensor? Why not just use Images?
\end{itemize}

\section{Challenges of the Research}


% 2. Chapter: ''Fundamentals Blocks''
\chapter{Fundamental Blocks of the Research} \label{chapter_two}


\section{Image Stabilisation}
What is image stabilisation?

% Start: Image Stabilisation
\subsection{Types of Image Stabilisation}
\subsubsection{Hardware Image Stabilisation}
Compensate camera movements with hardware. Expensive.

\subsubsection{Digital Image Stabilisation}
Compensate movements in post-processing. Economical.

\subsection{How Digital Image Stabilisation works}
We are interested in DIS so we will introduce the fundamentals.

% Start: IMU Sensors
\section{Inertial Measurement Unit Sensors}
\begin{itemize}
\item What are IMU Sensors?
\item What do they measure?
\item Applications?
\item Why use them for our application?
\item Advantages and Drawbacks
\end{itemize}

\subsection{Pose Estimation with IMUs}
How to measure pose with help of IMU Sensors?

\subsection{Noise Models}
About Noise in IMU Sensors

\subsection{Challenges in Working with IMU}
Various Challenges of working with IMUs.

% Start: Neural Networks
\section{Neural Network Architectures Explained}
We will be using a lot of Neural Networks. The basic functioning of them is explained here.

\subsection{Multi Layer Perceptron}
Most basic ANN.

\subsection{Convolutional Neural Networks}
More advanced neural networks.

\subsection{Recurrent Neural Networks}
Neural networks with memory.

\subsection{Attention Mechanism}
Talk about attention in Neural Networks.

\subsection{Transformer Networks}
How self attention is used in Neural Networks.

% 3. Chapter: ''State of Research''
\chapter{Various Methods and Approaches} \label{chapter_three}

In the third chapter, you present your own work. This chapter is the most important one. Give definitions for concepts, present algorithms in an abstract way and relate your work to concepts introduced in chapter two.
% 4. Chapter: ''Methods''
\chapter{Research Methodology} \label{chapter_four}

The fourth chapter is typically named {\em Realisation and Evaluation}. Here you give the most important technical details (more details are given in the appendix). Then you evaluate your approach -- typically by test runs. Describe the data used for the tests carefully. Give meaningful graphs. The fourth chapter together with the appendix should give all information which is necessary that somebody else can do the same test runs as you did.
% 5. Chapter: ''Realisation and Evaluation''
% \chapter{Conclusion} \label{chapter_five}

In this work, a novel video stabilization algorithm was developed that uses IMU data for camera motion estimation using neural networks. The videos were captured from a modified GoPro Hero 10 with a low field-of-view and high focal length lens. The camera had oscillations associated with it resulting in deterioration of video quality and visual discomfort for the viewer.

The video was digitally stabilized in real-time using only IMU data for camera motion estimation. Pose estimation with IMUs using classical algorithms was not feasible as the estimated pose drifts from the actual trajectory due to the noise present in IMU readings. This drift renders the video stabilization algorithm useless.

To eliminate drift and estimate the pose with high accuracy, data driven approaches were evaluated. They outperformed their classical predecessors and proved insusceptible to drift. Various neural network architectures were trained and evaluated for this purpose. Transformer networks proved to have the best performance for this use case as they had significantly lower error in stabilization trajectory estimation.

To train and evaluate these neural networks a huge amount of data is required. Thus, both real and simulated data was generated. Simulated data was generated based on camera oscillation characteristics. This allowed for easy generation of huge amount of data for training. Another benefit that the simulated data provided was that it was augmented with different IMU noise characteristics that make the neural networks robust towards noise and generalize well.

The output from the neural networks had sharp fluctuations and an exponential moving average filter was used to smooth it out which resulted in minimising the jitter and further improving the video stabilization quality. Finally, the run-time analysis of the deployed models was done to make sure the inference latency is in the allowed range.

The developed video stabilization algorithm was successful and generated highly accurate results. The stabilized videos are of good quality and suffered from minimal loss of visual content. 

\subsubsection{Future Work}
Although the developed image stabilization algorithm performs well, there are certain areas which require improvement. 

The algorithm currently only works at microscopic levels and needs to be adapted for high field of view and low focal length camera setups. This will require many changes in the warping grid estimation for stabilization. Current warping grid estimation requires depth to scene distance and this information is difficult to obtain in high field-of-view camera setups. The depth to scene information may be estimated using modern neural network based computer vision techniques or by using range finder sensors like LiDAR which comes installed with high end mobile devices.

The algorithm requires huge amount of data for training and evaluation and generating good data is a very expensive and time consuming process. To overcome this, semi-supervised or unsupervised learning techniques can be explored.

The developed pose estimation algorithm is accurate and can be further adapted for pose estimation in robotics and virtual reality. The transformer network performance is great and can yield good results for absolute pose estimation using only IMUs which is a huge challenge in the fields of robotics and virtual reality.

% The chapters Introduction and Conclusions have NO sections
% For the other chapters:
% Each sectioning (chapter, section, subsection, ...) should have NONE
% or AT LEAST TWO sub parts!
%\section{First Section}
%\subsection{First Subsection}
%\subsubsection{First Subsubsection}
%\paragraph{First Paragraph.} Some content for the first paragraph. 
%\subparagraph{First Subparagraph.} Some content for the first subparagraph. 
% Please note the dot after the paragraph and subparagraph heading. This is not accidently.

%Use bibtex for references to literature. 
%The most references will be in the second chapter.

% Make the references section
\bibliography{references} % According to the name of your bib file

% Start the appendix
\appendix
%%%%%%%%%%%%%%%%%%%%%%%%%%%%%%%%%%%%%%%%%%%
%  Place appendix content here            %
%  or use \input{appendix/appendixname}   %
%%%%%%%%%%%%%%%%%%%%%%%%%%%%%%%%%%%%%%%%%%%
\chapter{First Appendix Chapter}
\section{First Appendix Section}
\subsection{First Appendix Subsection}
\subsubsection{First Appendix Subsubsection}

\cleardoublepage

%----------------------------------------------------------------------------------------
%	DECLARATION PAGE
%---------------------------------------------------------------------------------------

% TODO uncomment the next three lines if your thesis is wirtten in german 
% \chapter*{Eidesstattliche Erklärung}
% \markboth{Eidesstattliche Erklärung}{Eidesstattliche Erklärung}
% \addcontentsline{toc}{chapter}{Eidesstattliche Erklärung}

% TODO comment the next three lines if your thesis is written in german 
\chapter*{Declaration of Authorship}
\markboth{Declaration of Authorship}{Declaration of Authorship}
\addcontentsline{toc}{chapter}{Declaration of Authorship}
% TODO Change the declaration according as needed. 

I hereby declare that I have written the above {\thesistype} thesis report independently and that I have not used any sources or aids other than those indicated. 

\bigskip

\begin{tabular}{@{}l@{}}
  Jena, Germany \rule[-0.8em]{7em}{0.5pt}\\[2ex]
  ~
\end{tabular}
\hspace{\fill}%
\begin{tabular}{@{}c@{}}
  \rule[-0.8em]{19em}{0.5pt}\\[2ex]
  {Ibad Rather}
\end{tabular}\hspace{\fill}

% Finish content and document
\end{document}
