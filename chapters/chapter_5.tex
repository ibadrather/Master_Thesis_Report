\chapter{Results and Discussion} \label{chapter_five}

We had a camera with a small FOV high focal length lens which kept vibrating after its rig was subjected to movement. The vibrations were very small, with an average translational displacement of around 2mm. This caused huge pixel shifts which deteriorated the quality of video recordings. Our goal was to stabilize the video so that there is no visual discomfort for the viewer and the quality of the scene is preserved.
    
Digital Image stabilization was chosen to stabilize the video for our use case as hardware stabilization is expensive and bulky with many moving parts. An MEMS IMU was chosen for camera motion estimation due to its low price and high reliability. However, the use of MEMS IMUs presents a huge challenge in accurate pose estimation as the noise characteristics of IMU, along with the use of classical inertial navigation algorithm, makes it impossible to accurately estimate pose. The use of advanced state estimation algorithms was also futile.

Data-driven approaches like RoNIN(CNN and ResNet) were tested to estimate camera pose accurately and showed more promising results than classical approaches. More advanced attention based neural network architectures were used (CNN-Transformer) which significantly improved the accuracy of stabilization trajectory estimation. Both real world and simulated data was collected to train the neural networks and test the image stabilization algorithm.

All these steps resulted into good quality IMU based digital video stabilization. The networks were able to accurately estimate the stabilization trajectory (camera pose) and did not suffer from drift. However, outputs from the neural networks were not smooth and caused some jitter in the stabilized video. Thus, exponential moving average filter was used on the outputs to smooth out the predicted trajectory and further improve the quality of video stabilization.

\subsubsection{Future Work}
\begin{itemize}
    \item Adapt this stabilization technique to high field of view and low focal length camera setups.

    \item Explore lean neural network architectures to obtain similar results with low latency networks that are deploy-able on cheaper hardware.

    \item Figure out way to use semi-supervised or unsupervised learning techniques to eliminate the hassle of ground truth generation.
    
\end{itemize}
