\chapter{State of Research} \label{chapter_three}

Previous chapter laid out the foundations for image stabilization and motion estimation. We also made the case for the techniques and algorithms chosen for this work. Now, this chapter includes some of the state-of-the-art methods for digital image stabilization and pose estimation. 

\section{Digital Image Stabilization}



\section{Pose Estimation}
\label{sec:sota_pose_est}
Classical algorithms fail to accurately estimate the pose using MEMS IMU over a long period of time. With the advancement of artificial neural networks (ANNs) and their architectures new data driven techniques have been developed for this purpose. These data-driven techniques have far outperformed their classical predecessors and are becoming a standard for this purpose. But with all these benefits they are computationally very expensive and have very high edge hardware requirements. 

\subsection{RIDI: Robust IMU Double Integration}
RIDI was the first data-driven neural network approach which tackles the challenges of double integrating accelerometer readings. The algorithm regresses a velocity vector from the history of linear accelerations and angular velocities (IMU windows - discussed in section \ref{sec:data_structure}), then corrects low-frequency bias in the linear accelerations, which are integrated twice to estimate position \citep{yan2018ridi}. The analysis of results manifested that this algorithm outperformed the heuristic-based methods and even came closer to visual inertial navigation \citep{yan2018ridi}. The issue with this is it can only regress position and orientation estimation needs to be achieved using classical algorithms.

\subsection{RoNIN: Robust Neural Inertial Navigation in the Wild}
Developed by the same team as RIDI, RoNIN uses more advanced neural network architectures like ResNet, LSTM and TCN as its backbone to regress a velocity vecotor given a window of IMU samples \citep{herath2020ronin}. Using these architectures and increasing the quality and size of data RoNIN outperforms RIDI and sets a new benchmark for inertial navigation. Our methodology is hugely infleuenced by this technique and is discussed in chapter \ref{chapter_four} of this report.

\subsection{TLIO: Tightly Learned Inertial Odometry}
This combines both classical and learn-based approaches by using an Extended Kalman Filter (EKF) and deep neural network (DNN). EKF is responsible for prediction step while outputs (displacement and uncertainty) from the neural network is used as measurement update. The filter estimates rotation, velocity, position and IMU biases at IMU rate \citep{hol2009tightly}. We could not reproduce the same results as the research claimed using our datasets.

\subsection{CTIN: Robust Contextual Transformer Network for Inertial Navigation}
This algorithm takes advantage of the recent advances in deep learning by using Multi-Headed-Attention (MHA) mechanism introduced in \citep{vaswani2017attention}. It uses a ResNet based encoder enhanced by local and global MHA to capture spatial contextual information from IMU measurements \citep{rao2022ctin}. CTIN is claimed to be robust and to outperform state-of-the-art models but cannot be evaluated as it is not available publicly.



