\chapter{Introduction} \label{chapter_one}

Cameras have become an integral part of our lives. We use cameras regularly, be it a smartphone camera, a hobbyist camera setup or professional cameras. Our demands for high quality media from cameras is also increasing constantly. We want sharper, color accurate, low lighting bright photographs. The videos are expected to capture scenes with as much realism as possible and we don't want too many movements in them irrespective of the way we capture them. If we are making a video capturing a beautiful sunrise or a landscape while walking, we do not want movements associated with walking in our video. We want it to be as smooth as possible. This is where \textit{Image Stabilization} comes into play. Its goal being producing a stable video irrespective of the camera movements during capture. Stabilization is a very important quality metric for videos. A video may have all the aspects of photography correct but if it has too many sudden movements it would not look good.

% Motivation
\section{Motivation}

Image stabilization is a key technology to produce  good quality photographs and videos. Be it a 250,000 Euros television broadcast camera or a 100 Euros hobbyist camera, stabilization of some sort is necessary. That is why image stabilization exists in every camera and smartphone these days. There even exists a series of product lines from many big companies like DJI just for this specific reason; to stabilize the image (video). These products can cost between less than 100 Euros to upwards of 1000 Euros or more for professionals. This clearly indicates that the need for stabilization is there and constant efforts are being put to make it better.

On microscopic level, the stabilization of videos becomes even more important as even small movements can cause large pixel shifts in the image plane. This can be seen by zooming in on your camera while trying to keep it stable. The effect of magnification becomes more evident and it will become increasingly more difficult to keep the image smooth and stable as the movements are magnified as well. This will deteriorate the video quality and is very important to deal with. 

Microscopic movements are the scope of this work. A modified GoPro Hero 10 (figure \ref{fig:gopro_hero10}) camera is used as it allows to extract synchronized sensor and video data. The wide angle fisheye lens of the GoPro was exchanged with a narrow field-of-view lens that has a very large focal length. Switching the lens resulted in the default stabilization provided by GoPro (HyperSmooth) to not work anymore. The goal of this work is to explore different methods to obtain a stabilized video stream for large focal length lenses.

\begin{figure}[H]
    \centering
    \includegraphics[scale=0.1]{images/fig_chapter1/gopro_hero10.png}
    \caption{GoPro Hero 10}
    \label{fig:gopro_hero10}
\end{figure}


%\begin{itemize}
%\item Why am I doing this?
%\item Why is this relevant?
%\item What concrete problem do we want to solve?
%\item Is this problem big enough?

%\item Why is this exciting?
%\item What makes this problem challenging?
%\item Why an IMU Sensor? Why not just use Images?
%\end{itemize}

%How many times have to captured a video of something and when you looked at it later you found out that the video  is shaky? Image stabilization is inherent in producing good video quality. Be it a 250,000 Euro television broadcast camera or a 100 euro hobbyist camera, stabilization of some sort is necessary.


\section{Challenges}

\begin{itemize}
\item We have a camera with small FOV high focal length lens which keeps vibrating on its rig after the rig is moved around the scene.

\item Vibrations are very small with a mean amplitude of 1.8mm but cause huge pixel shifts in the video recordings thus deteriorating the quality.

\item Our goal is to stabilize the video so that there is no visual discomfort for the viewer and quality of the scene is preserved.

\item We want to use an Inertial Measurement Unit (IMU) sensor to track these vibrations and stabilize the video based on it.

\item Sensors are inherent to noise and IMUs being no different present a lot of challenges in accurate camera motion estimation.

\item There exist IMU sensors with better noise characteristics, however, those are very expensive (> 1000 Euros) and are difficult to buy as they usually have a tactical grade placing them under export restrictions. Hence, in this work a low-cost consumer MEMS (Microelectromechanical systems) IMU is used.

\item Using IMUs for both rotational and translational motion estimation for image stabilization present many challenges including:
\begin{itemize}
    \item Accurate motion estimation over a longer period of time is not possible with classical algorithms.
    
    \item The positional \textbf{drift} in translational motion estimation increases non-linearly with time which makes the image stabilization useless.

    \item This results in to output stabilized video quality being worse
    
\end{itemize}

\item Working with this drift is especially not possible in this case as we have a accuracy requirement of about 0.5 mm for the stabilization to work effectively.

\item IMUs have been used for position estimation for a long time using the classic Kalman Filter and its variants. They produce acceptable results for some cases but would not work in our case as they are susceptible to strong drift without absolute position updates. IMUs are also generally coupled with cameras, GPS and Magnetometers to improve the accuracy but it is not possible for this use case.

\item Irrespective of these pitfalls, the use of IMU is necessary for estimating these small motions. To tackle these challenges the following contributions are presented in this thesis:

\begin{itemize}
    \item Analysis of real motion characteristics and building a simulation based on that analysis to generate huge amount of data.

    \item Generating good data for data-driven approaches is very difficult, so, techniques for proper domain randomization are explored.

    \item Exploring the possible use of various neural network architectures to learn motion characteristics of these vibrations and are not susceptible to the drift.
\end{itemize}

\end{itemize}

There are a lot of challenges in this work to overcome and that makes this exciting. In the next chapter , fundamental of image stabilization and pose-estimation will be discussed. It will also cover some possible solutions and their pitfalls and advantages.
