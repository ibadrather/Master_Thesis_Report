\chapter{Introduction} \label{chapter_one}

Cameras have become an integral part of our lives in the current world  \citet{ibad2022}. People around us use cameras regularly, be it their smartphone camera, a hobbyist camera setup or professional cameras. Also everyone want a cameras that produce the good picture quality for as cheap as possible. Researchers are constantly working towards making this technology better while also trying to make it cheaper.

There are many challenges involved in making the picture quality better. For example you may have noticed if you are taking a photograph and the camera moves while taking the photograph or a video, the photo does not look good. This problem is solvable if you have a tripod or expensive motorised stabilisers like Gimbals. But as you may have understood that is not always feasible, economically and logistically.

These problems are aggravated on microscopic levels or if you are using a zoom lens, in that case even very small movements can cause the picture quality to degrade. That is exactly the problem I am facing for this research. I am working with a GoPro Hero 10 but with a different lens than fitted from the factory one. I switched the lens to a Macro or Zoom Lens with less field of view (FOV) and different focal length. Switching the lens resulted in the default stabilising provided by the GoPro not working.

As a part of this research work, I will explore different ways in which I can stabilise the video feed from the camera. Before going into depth, I need to first talk about a lot of important things. First of all let us talk about why am I even doing this? Is this even a real problem that needs to  be solved?

% Motivation
\section{Motivation}
\begin{itemize}
\item Why are we doing this?
\item Why is this relevant?
\item Why is this exciting?
\item What concrete problem do we want to solve?
\item Is this problem big enough?
\item What makes this problem challenging?
\item Why an IMU Sensor? Why not just use Images?
\end{itemize}

\section{Challenges of the Research}

