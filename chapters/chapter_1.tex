\chapter{Introduction} \label{chapter_one}

Cameras have become an integral part of our lives in the current world. We use cameras regularly, be it a smartphone camera, a hobbyist camera setup or professional cameras. Our demands for high quality media from cameras is also increasing constantly. We want sharper, color accurate, low lighting bright photographs. The videos are expected to capture scenes with as much realism as possible and also we don't want too many movements in them irrespective of the way we capture them. If we are making a video capturing a beautiful sunrise or a landscape while walking, we do not want movements associated with walking in our video. We want it to be as smooth as possible. This is where \textit{Image Stabilization} comes into play. Its goal being producing a stable video irrespective of the camera movements during capture. Stabilization is a very important quality metric for videos. A video may have all the aspects of photography correct but if it has too many sudden movements it would not look very good.



% Motivation
\section{Motivation}

Image stabilization is a key technology to produce  good quality photographs and videos. Be it a 250,000 Euros television broadcast camera or a 100 Euros hobbyist camera, stabilisation of some sort is necessary. That is why some sort of image stabilization exists in every camera and smartphone these days. There even exists a series of product lines from many big companies like DJI just for this specific reason; to stabilise the image (video). These products can cost between 80 Euros for hobbyists to 1300 Euros or more for professionals. This clearly indicates that the need for stabilization is there and constant efforts are being put to make it better.

On microscopic level, the stabilization of videos becomes even more important as even the small movements can cause large pixel shifts on the image levels. You can see this by zooming in from your camera on any object and try to keep it stable. Even if you feel you are holding the camera stationary, there will still be a lot of movements in the video. This happens because at a microscopic level the movements are also magnified. This can cause the video quality to deteriorate and is very important to deal with. 

This is exactly the problem I am facing for this research. I am working with a GoPro Hero 10 but with a different lens than fitted from the factory one. I switched the lens to a Macro or Zoom Lens with less field of view (FOV), higher magnification and a different focal length. Switching the lens resulted in the default stabilising provided by the GoPro not working as it was designed for a different camera setup. Now, my goal for this research work is to explore different ways in which I can stabilise the video from the camera.

%\begin{itemize}
%\item Why am I doing this?
%\item Why is this relevant?
%\item What concrete problem do we want to solve?
%\item Is this problem big enough?

%\item Why is this exciting?
%\item What makes this problem challenging?
%\item Why an IMU Sensor? Why not just use Images?
%\end{itemize}

%How many times have to captured a video of something and when you looked at it later you found out that the video  is shaky? Image stabilisation is inherent in producing good video quality. Be it a 250,000 Euro television broadcast camera or a 100 euro hobbyist camera, stabilisation of some sort is necessary.


\section{Challenges of the Research}
To solve the problem or to stabilize the video as a part of this research I will be using an IMU sensor along with the image sensor. The reasons for choosing this are discussed in the next section of this report. Doing this is very challenging and will require the use of some of the state of the art neural network based techniques present today. Some of the challenges that were faced for this research and the precision requirements are as follows:

\begin{itemize}
\item Sensors are inherent to noise and this especially makes working with an IMU difficult because for position estimation we need to double integrate the signal. This double integration of the noisy signal causes a \textbf{drift} which is very undesirable for any image stabilization algorithm. 
\item Working with this drift is especially not possible in this case as we have a precision requirement of about 0.1 mm for the stabilization to work effectively.
\item There is a possibility of buying IMU sensors with less noise characteristics but those sensors are very expensive in the order of thousands of Euros. Using these sensors will be very expensive. We want to achieve the stabilisation by using a sensor in the price range of around 10 Euros. 
\item IMUs have been used for position estimation for a long time using the classic Kalman Filter and its variants. They produce acceptable results for some cases but would not work in our case. IMUs are also generally coupled with cameras, GPS and Magnetometers to improve the precision but it is not possible for this use case.
\item  To over some of the challenges I am using neural networks. The use of neural networks creates a new problem of data generation for training and testing. Neural network implementations for any task require a huge amount of data and the ground truth associated with it.
\item Creating this data is very challenging and a lot of care needs to be taken about domain randomisation so the network generalizes well.
\end{itemize}
There are a lot of challenges that need to be dealt with to come up with a good solution to this problem and that makes this problem so exciting. In the next chapter of this report, I will discuss how image stabilization works fundamentally. And then methods that are available and why I have chosen the techniques and tools for this research.

